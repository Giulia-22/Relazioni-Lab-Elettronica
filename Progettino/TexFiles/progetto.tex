
\documentclass{report}

\usepackage[utf8]{inputenc}
\usepackage[italian]{babel}
\usepackage{import}
\usepackage{todonotes}
\usepackage{color}
\usepackage{rotating}
\usepackage[hidelinks]{hyperref}
\usepackage{url}
\usepackage{pdfpages}
\usepackage{siunitx}
\usepackage{pdflscape}
\usepackage{subfig}
\usepackage[euler]{textgreek}
\usepackage{mhchem}

\usepackage{enumerate} 
\usepackage{amsmath}
\usepackage{amsfonts}

\usepackage[signatures,swapnames,sans]{frontespizio}

\usepackage{geometry}
\geometry{portrait, margin=3cm}
\usepackage{siunitx}
\usepackage{booktabs}

\renewcommand*\figurename{Figura}

\newcommand{\sub}[1]{\textsubscript{#1}}
\newcommand{\super}[1]{\textsuperscript{#1}}
\newcommand{\parallelsum}{\mathbin{\!/\mkern-5mu/\!}}

\newcommand{\Fig}[0]{Fig.}

\usepackage{titlesec}

\titleformat{\chapter}{\normalfont\huge}{}{20pt}{\huge\bfseries}

\linespread{1.3}

\begin{document}
\addtocounter{chapter}{+5}
	\begin{frontespizio}
		\Margini{3cm}{3cm}{3cm}{3cm}
		\Universita{Bergamo}
		\Logo[43.332mm]{unibg-mark}
		\Divisione{Scuola di Ingegneria}
		\Corso[Laurea Magistrale]{Ingegneria Informatica}
		\Titolo{Laboratorio di Elettronica}
		\Sottotitolo{Relazione progetto circuito}
		\Punteggiatura{}
		\NRelatore{Prof.}{Prof.}
		\Relatore{Luigi Gaioni}
		\Candidato[1058231]{Giulia Allievi}
		\Candidato[1059640]{Martina Fanton}
		\Annoaccademico{2022--2023}
		\begin{Preambolo*}
			\usepackage[italian]{babel}
			\usepackage[T1]{fontenc}
			\usepackage[utf8]{inputenc}
			\usepackage{microtype}
			\usepackage{lmodern}
			\graphicspath{{img/}}
			
			\renewcommand{\frontinstitutionfont}{\fontsize{14}{17}\bfseries\scshape}
			\renewcommand{\fronttitlefont}{\fontsize{17}{21}\bfseries\scshape}
			\renewcommand{\frontfootfont}{\fontsize{12}{14}\bfseries\scshape}
		\end{Preambolo*}
	\end{frontespizio}

%----------------------------------------------------------------------------------------
%	PAGINA BIANCA
%----------------------------------------------------------------------------------------
\newpage
\null
\thispagestyle{empty}
\newpage

\chapter{Relazione progetto circuito}
\section*{Introduzione}
Il progetto richiede di realizzare un circuito che, superata una temperatura di riferimento, generi un allarme luminoso lampeggiante. Il sistema deve essere automatico, reversibile e realizzato hardware, senza avere a disposizione microcontrollori. Si hanno a disposizione:
\begin{itemize}
\item un termistore NTC;
\item un LED rosso;
\item un comparatore;
\item un timer 555;
\item componenti passivi.
\end{itemize}
La temperatura di riferimento è \SI{25}{\degree C}, a questa temperatura la resistenza del termistore NTC è di \SI{1}{k\ohm}.
% modifica per non copiare da slide

\newpage
\section{Progettazione del circuito}
Per progettare il circuito, progetteremo e dimensioneremo separatamente la rete del termistore e la rete oscillante, quindi integreremo le sue sottoreti per ottenere il progetto del sistema finale.
\subsection{Progettazione della rete oscillante}
Inizialmente progettiamo la rete oscillante. Configuriamo il timer 555 in modo tale che funzioni in modalità astabile. Lo schema scelto è mostrato in figura \ref{figura:schema555}, i pin sono collegati in questo modo:
\begin{itemize}
\item PIN 1, è il terminale di \textit{ground}, perciò è collegato a massa;
\item PIN 2, è il terminale di \textit{trigger}, è cortocircuitato con il PIN 6;
\item PIN 3, è l'\textit{uscita}, a cui sarà collegato il LED tramite una resistenza;
\item PIN 4, è il terminale di \textit{reset}, servirà per gestire il collegamento alla rete che pilota il termistore;
\item PIN 5, è il terminale di \textit{control voltage}, non lo utilizziamo, è collegato a massa tramite una capacità di filtraggio ($\mathrm{C_1}$);
\item PIN 6, è il terminale di \textit{threshold}, gestisce la carica della capacità $\mathrm{C_2}$ attraverso la resistenza $\mathrm{R_1}$;
\item PIN 7, è il terminale di \textit{discharge}, pilota la scarica della capacità $\mathrm{C_2}$ attraverso la resistenza $\mathrm{R_2}$;
\item PIN 8, è il terminale di \textit{alimentazione}. 
\end{itemize} 
Tra i pin 6 e 7 viene collegato un diodo (l'anodo è collegato al pin 7 mentre il catodo al pin 6), la sua funzione è quella di bypassare la resistenza $\mathrm{R_2}$ nella fase di carica del condensatore, in modo tale da ottenere un oscillatore con duty cycle variabile da 0\% a 100\%. 
\begin{figure}[h!]
	\centering
	\includegraphics[height=6.9cm]{immagini/schema555}
	\caption{Schema della rete oscillante.} 
	\label{figura:schema555}
\end{figure}
\\Successivamente, dimensioniamo la rete oscillante. Le grandezze da dimensionare sono:
\begin{itemize}
\item tensione di alimentazione $V_A$;
\item capacità $C_1$ e $C_2$;
\item resistenze $R_1$, $R_2$ e $R_3$.
\end{itemize}
Per primo, scegliamo il valore che deve avere la tensione di alimentazione $V_A$. Dal \textcolor{blue}{\underline{\href{https://www.ti.com/lit/ds/symlink/lm555.pdf?ts=1667144089940&ref_url=https\%253A\%252F\%252Fwww.ti.com\%252Fproduct\%252FLM555}{datasheet}}} del timer 555, vediamo che il componente deve essere alimentato con una tensione compresa fra \SI{4.5}{\volt} e \SI{16}{\volt}: dato che il progetto della rete del termistore (sezione \ref{rete_termistore}) prevederà di utilizzare un OPAMP, scegliamo un valore compatibile anche con questo componente, in modo tale da avere un'unica tensione di alimentazione per il circuito finale. Dal \textcolor{blue}{\underline{\href{https://www.ti.com/lit/ds/symlink/ua741.pdf?ts=1672216941275&ref_url=https\%253A\%252F\%252Fwww.ti.com\%252Fproduct\%252FUA741}{datasheet}}} del \textmu A741, un amplificatore operazionale \textit{general purpose}, sappiamo che dobbiamo scegliere una tensione duale o singola compresa fra \SI{-18}{\volt} e +\SI{18}{\volt}. Di conseguenza, dobbiamo scegliere un valore di  $V_A$ di circa \SI{10}{\volt}: visto che nella realtà il circuito non funzionerà con un alimentatore da banco, ma con delle batterie, scegliamo di alimentare i componenti attivi con una tensione da \SI{9}{\volt}, così da poter utilizzare queste batterie. \par
Successivamente, scegliamo i valori delle capacità. La capacità $\mathrm{C_1}$ serve per filtrare il segnale di massa da eventuali disturbi, il valore consigliato dal datasheet è di \SI{0.01}{\mu\farad}, perciò $\displaystyle{C_1=\SI{0.01}{\mu\farad}}$. La capacità $\mathrm{C_2}$ regola il periodo di oscillazione, scegliamo una capacità da \SI{200}{\mu\farad}. \par
Dimensionate le capacità, scegliamo i valori che devono avere le resistenze $R_1$ e $R_2$. Per fare ciò, decidiamo prima per quanto tempo il LED deve rimanere acceso e per quanto spento, da questi due intervalli di tempo ricaveremo i valori delle due resistenze. Le formule che descrivono queste due grandezze sono:
\\$$t_{low}=\ln2\cdot R_2\cdot C_2\indent\indent\indent \mathrm{e}\indent\indent\indent t_{high}=\ln2\cdot R_1\cdot C_2$$
Vorremmo che il LED resti spento per \SI{1}{\second} e acceso per \SI{2}{\second}, dunque $\displaystyle{t_{low}=\SI{1}{\second}}$ e $\displaystyle{t_{high}=\SI{2}{\second}}$. Perciò, dalle formule inverse si ricavano i valori delle resistenze $R_1$ e $R_2$:
\\$$R_2 = \frac{t_{low}}{\ln2\cdot C_2}=\frac{\SI{1}{\second}}{0.693\cdot\SI{200}{\mu\farad}}=\SI{7.213}{k\ohm}\simeq\SI{7.5}{k\ohm}$$
$$R_1 = \frac{t_{high}}{\ln2\cdot C_2}=\frac{\SI{2}{\second}}{0.693\cdot\SI{200}{\mu\farad}}=\SI{14.427}{k\ohm}\simeq\SI{15}{k\ohm}$$
Con i valori delle resistenze approssimati ai valori reali, i due tempi risultano $\displaystyle{t_{low}=\SI{1.04}{\second}\mathrm{\;e\;}t_{high}=\SI{2.08}{\second}}$, perciò i calcoli risultano in accordo con quanto scelto. \par
L'ultima resistenza da dimensionare è $\mathrm{R_3}$. Questa resistenza ha lo scopo di ridurre la corrente che fluisce nel LED, altrimenti si rischia di bruciarlo. Di solito, in questi dispositivi circola una corrente di 15\SI{-20}{m\ampere}, per il dimensionamento ipotizziamo che nel diodo circoli una corrente pari a \SI{20}{m\ampere}, mentre la caduta di tensione ai capi di un LED di colore rosso è di \SI{1.8}{\volt}. Per ricavare il valore di $\mathrm{R_3}$, basta utilizzare la legge di Ohm:
$$\indent i_R = i_D\indent\rightarrow\indent \frac{V_{out}-V_D}{R_3}=i_D\indent\rightarrow\indent\frac{\SI{9}{\volt}-\SI{1.8}{\volt}}{R_3}=\SI{20}{m\ampere}$$
$$\Rightarrow\indent R_3=\frac{(9\SI{-1.8}{)\volt}}{\SI{20}{m\ampere}}=\SI{360}{\ohm}$$
Per $\mathrm{R_3}$ scegliamo una resistenza da \SI{500}{\ohm}, di conseguenza nel diodo fluirà una corrente di circa \SI{15}{m\ampere}. \par 
Nell'immagine in figura \ref{figura:dim555} è mostrato il sottosistema della rete oscillante con i valori scelti per ogni componente. 
\begin{figure}[h!]
	\centering
	\includegraphics[height=6.9cm]{immagini/dim555}
	\caption{Schema della rete oscillante dimensionata.} 
	\label{figura:dim555}
\end{figure}
\\Nella figura \ref{figura:555out} vengono invece mostrati i grafici che si ottengono in uscita (corrente che fluisce nel LED) quando il circuito si trova nello stato di allarme e di riposo. \par
Nel primo caso, la temperatura sarà superiore a \SI{25}{\degree}C, quindi il sistema si troverà nello stato di allerta. Il circuito si comporterà come un oscillatore, perciò, terminato il transitorio iniziale, l'uscita resta alta per circa due secondi e bassa per circa un secondo. Le imprecisioni su $t_{high}$ sono dovute al fatto che nei calcoli è stata trascurata la caduta di tensione data dal diodo D1, perciò quest'intervallo di tempo è di poco superiore rispetto a quanto dimensionato. Dalle misure con i cursori, questa differenza risulta essere pari a \SI{0.36}{\second}. \par
Nel secondo caso invece, il sistema registrerà una temperatura inferiore a \SI{25}{\degree}C, di conseguenza nel LED non deve fluire corrente perché deve rimanere spento, dato che il sistema non è nello stato di allarme.
\begin{figure}[h!]
	\centering
	\includegraphics[height=6cm]{immagini/555on}
	\includegraphics[height=6cm]{immagini/555off}
	\caption{Grafico dell'uscita quando il sistema è nello stato di allarme (sopra) e riposo (sotto).} 
	\label{figura:555out}
\end{figure}
\todo{spostare foto 6.3 prima di "Nel primo caso..." e metterle sulla stessa riga?}
\subsection{Progettazione della rete del termistore}\label{rete_termistore}
Per gestire le due situazioni descritte alla fine della sezione precedente, dobbiamo pilotare opportunamente il PIN 4, ovvero il \textit{reset}, del timer 555. In particolare, se la temperatura è inferiore a \SI{25}{\degree}C, sul PIN 4 deve essere applicato un segnale di tensione che corrisponde al livello logico basso (\SI{0}{\volt}), perché vogliamo che anche l'uscita si trovi al livello logico basso,  in modo tale che il LED non si accenda. Al contrario, se la temperatura è superiore alla temperatura di riferimento, al reset deve essere applicato un segnale corrispondente al livello logico alto (\SI{9}{\volt}), così da far lavorare il timer 555 in modalità astabile e pilotare l'accensione e lo spegnimento del LED secondo i tempi scelti nella sezione precedente. \par
\noindent Queste scelte sono dovute al fatto che il timer 555 viene resettato quando si applica un impulso negativo sul PIN 4 (fonte: \textcolor{blue}{\underline{\href{https://www.ti.com/lit/ds/symlink/lm555.pdf?ts=1667144089940&ref_url=https\%253A\%252F\%252Fwww.ti.com\%252Fproduct\%252FLM555}{datasheet}}}), e questa transizione la otteniamo se utilizziamo le tensioni come appena descritto. \par
Il segnale che pilota il reset è dato dall'OPAMP, che sarà utilizzato in configurazione di comparatore: per la nostra applicazione, è necessario configurare gli ingressi invertente e non invertente per fare in modo che la condizione $V^+>V^-$ si verifichi quando la temperatura è superiore a \SI{25}{\degree}C, così che l'uscita dell'amplificatore operazionale sia alta, il viceversa deve invece verificarsi quando la temperatura è minore del valore critico. \par
Per decidere come dimensionare gli ingressi, dobbiamo prima capire il comportamento elettrico del termistore al variare della temperatura. Un termistore NTC è schematizzabile come una resistenza variabile, il cui andamento in funzione della temperatura (in kelvin) è di tipo esponenziale:
$$ R_T = R_0\cdot e^{B\bigl(\frac{1}{T}-\frac{1}{T_0}\bigr)}$$
Dalla specifica, sappiamo che $R_0$ vale \SI{1}{k\ohm} quando $T_0 = \SI{25}{\degree C}=\SI{298}{\kelvin}$. Il parametro $B$ varia in base al termistore scelto e in particolare per un termistore NTC con le nostre caratteristiche vale circa \SI{3000}{\kelvin}, come possiamo ricavare dal seguente \textcolor{blue}{\underline{\href{https://www.mouser.it/datasheet/2/240/Littelfuse_Leaded_Thermistors_Glass_Encapsulated_T-1315924.pdf}{datasheet}}}. Sostituendo alla relazione precedente le grandezze note, otteniamo:
$$R_T = 42.45\cdot 10^{-3}\;\SI{}{\ohm}\cdot e^{\bigl(\frac{\SI{3000}{\kelvin}}{T}\bigr)} $$
Dalla formula, verifichiamo che a \SI{25}{\degree C}=\SI{298}{\kelvin} la resistenza è effettivamente di \SI{1}{k\ohm}.
% prendo due valori (uno + e l'altro - di 25) e vedo comportamento -> r aumenta se t minore e viceversa
% procedo con dimensionamento partitore ingressi opamp


\newpage
\section{Simulazione del circuito}
Una volta integrate le due sottoreti del sistema nel circuito complessivo, abbiamo simulato quest'ultimo utilizzando due programmi differenti: LTSpice e Tinkercad.
% unione dei due sotto sistemi
\subsection{Simulazione con LTSpice}
\subsection{Simulazione con Tinkercad}
In Tinkercad per prima cosa abbiamo realizzato il circuito su una breadboard (visibile in figura \todo{inserire figura}) inserendovi i componenti dimensionati nella sezione precedente.
Poi per valutarne il comportamento, abbiamo simulato il circuito.



\end{document}