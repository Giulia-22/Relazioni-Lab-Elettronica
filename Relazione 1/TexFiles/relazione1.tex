
\documentclass{report}

\usepackage[utf8]{inputenc}
\usepackage[italian]{babel}
\usepackage{import}
\usepackage{todonotes}
\usepackage{color}
\usepackage{rotating}
\usepackage[hidelinks]{hyperref}
\usepackage{url}
\usepackage{pdfpages}
\usepackage{siunitx}
\usepackage{pdflscape}
\usepackage{subfig}
\usepackage[euler]{textgreek}
\usepackage{mhchem}

\usepackage{enumerate} 
\usepackage{amsmath}
\usepackage{amsfonts}

\usepackage[signatures,swapnames,sans]{frontespizio}

\usepackage{geometry}
\geometry{portrait, margin=3cm}
\usepackage{siunitx}
\usepackage{booktabs}

\renewcommand*\figurename{Figura}

\newcommand{\sub}[1]{\textsubscript{#1}}
\newcommand{\super}[1]{\textsuperscript{#1}}
\newcommand{\parallelsum}{\mathbin{\!/\mkern-5mu/\!}}

\newcommand{\Fig}[0]{Fig.}

\usepackage{titlesec}

\titleformat{\chapter}{\normalfont\huge}{}{20pt}{\huge\bfseries}

\linespread{1.1}

\begin{document}
\addtocounter{chapter}{+1}
	\begin{frontespizio}
		\Margini{3cm}{3cm}{3cm}{3cm}
		\Universita{Bergamo}
		\Logo[43.332mm]{unibg-mark}
		\Divisione{Scuola di Ingegneria}
		\Corso[Laurea Magistrale]{Ingegneria Informatica}
		\Titolo{Laboratorio di Elettronica}
		\Sottotitolo{Relazione esperienza di laboratorio}
		\Punteggiatura{}
		\NRelatore{Prof.}{Prof.}
		\Relatore{Luigi Gaioni}
		\Candidato[1058231]{Giulia Allievi}
		\Candidato[1059640]{Martina Fanton}
		\Annoaccademico{2022--2023}
		\begin{Preambolo*}
			\usepackage[italian]{babel}
			\usepackage[T1]{fontenc}
			\usepackage[utf8]{inputenc}
			\usepackage{microtype}
			\usepackage{lmodern}
			\graphicspath{{img/}}
			
			\renewcommand{\frontinstitutionfont}{\fontsize{14}{17}\bfseries\scshape}
			\renewcommand{\fronttitlefont}{\fontsize{17}{21}\bfseries\scshape}
			\renewcommand{\frontfootfont}{\fontsize{12}{14}\bfseries\scshape}
		\end{Preambolo*}
	\end{frontespizio}


%----------------------------------------------------------------------------------------
%	INTRO
%----------------------------------------------------------------------------------------
\section{Introduzione}
\section{Schema e analisi teorica}
\section{Misure e osservazioni}
\begin{table}
\begin{tabular}{|c|c|c|c|c|}
		\hline
		\textbf{Frequenza [{\unit{\hertz}}]} & \textbf{\boldmath$\displaystyle{V_{PP,in}}$ [\unit{\volt}]} & \textbf{\boldmath$\displaystyle{V_{PP,out}}$ [\unit{\volt}]} & \textbf{Guadagno} & \textbf{Sfasamento [\unit{\degree}]}\\
		\hline
		100 m & 0.488 & 4.804 & 9.84 & -179.3\\
		\hline
		500 m & 0.488 & 4.799 & 9.83 & -176.8\\
		\hline
		1 k & 0.485 & 4.779 & 9.85 & -173.4\\
		\hline
		5 k & 0.481 & 4.188 & 8.71 & -149.7\\
		\hline
		8.8 k & 0.484 & 3.388 & 7 & -133.9\\
		\hline
		10 k & 0.485 & 3.163 & 6.52 & -129.7\\
		\hline
		50 k & 0.483 & 0.833 & 1.72 & -96.3\\
		\hline
		100 k & 0.484 & 0.427 & 0.88 & -87.7\\
		\hline
		500 k & 0.487 & 0.943 & 1.94 & -58.6\\
		\hline
		1 M & 0.488 & 0.542 & 1.11 & -41.9\\
		\hline
		5 M & 0.477 & 0.621 & 1.3 & -35.9\\
		\hline
		10 M & 0.43 & 0.106 & 0.25 & -9.8\\
		\hline
	\end{tabular}
\end{table}


%----------------------------------------------------------------------------------------

\end{document}
