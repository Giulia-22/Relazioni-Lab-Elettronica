
\documentclass{report}

\usepackage[utf8]{inputenc}
\usepackage[italian]{babel}
\usepackage{import}
\usepackage{todonotes}
\usepackage{color}
\usepackage{rotating}
\usepackage[hidelinks]{hyperref}
\usepackage{url}
\usepackage{pdfpages}
\usepackage{siunitx}
\usepackage{pdflscape}
\usepackage{subfig}
\usepackage[euler]{textgreek}
\usepackage{mhchem}

\usepackage{multirow}

\usepackage{enumerate} 
\usepackage{amsmath}
\usepackage{amsfonts}

\usepackage[signatures,swapnames,sans]{frontespizio}

\usepackage{geometry}
\geometry{portrait, margin=3cm}
\usepackage{siunitx}
\usepackage{booktabs}

\renewcommand*\figurename{Figura}

\newcommand{\sub}[1]{\textsubscript{#1}}
\newcommand{\super}[1]{\textsuperscript{#1}}
\newcommand{\parallelsum}{\mathbin{\!/\mkern-5mu/\!}}

\newcommand{\Fig}[0]{Fig.}

\usepackage{titlesec}

\titleformat{\chapter}{\normalfont\huge}{}{20pt}{\huge\bfseries}

\linespread{1.3}


%% COMANDI UTILI
%\begin{table}[h]
%	\centering
%	\begin{tabular}{|c|c|c|}
%	\cline{2-3} 
%	\multicolumn{1}{c|}{} & \textbf{Valore nominale} & \textbf{Valore misurato}\\ 
%		%\hline
%		%{} & \textbf{Valore nominale} & \textbf{Valore misurato} \\ 
%		\hline
%		$\mathbf{R_1}$ & \SI{18}{k\ohm} & \SI{17.977}{k\ohm} \\ 
%		\hline
%		$\mathbf{R_2}$& \SI{1.8}{k\ohm} & \SI{1.815}{k\ohm} \\ 
%		\hline
%	\end{tabular}
%\caption{Misure delle resistenze utilizzate per il circuito.}
%\label{table:mis_res}
%\end{table}
%\begin{figure}[h!]
%\centering
%\includegraphics[height=6.5cm]{immagini/TEK00018}\\(a)\\[1ex]
%\includegraphics[height=6.5cm]{immagini/TEK00019}\\(b)
%\caption{Risposta del circuito con accoppiamento DC (a) e accoppiamento AC (b).}
%	\label{figura:accopp}
%\end{figure}

\begin{document}
\addtocounter{chapter}{+2}
	\begin{frontespizio}
		\Margini{3cm}{3cm}{3cm}{3cm}
		\Universita{Bergamo}
		\Logo[43.332mm]{unibg-mark}
		\Divisione{Scuola di Ingegneria}
		\Corso[Laurea Magistrale]{Ingegneria Informatica}
		\Titolo{Laboratorio di Elettronica}
		\Sottotitolo{Relazione esperienza di laboratorio 3}
		\Punteggiatura{}
		\NRelatore{Prof.}{Prof.}
		\Relatore{Luigi Gaioni}
		\Candidato[1058231]{Giulia Allievi}
		\Candidato[1059640]{Martina Fanton}
		\Annoaccademico{2022--2023}
		\begin{Preambolo*}
			\usepackage[italian]{babel}
			\usepackage[T1]{fontenc}
			\usepackage[utf8]{inputenc}
			\usepackage{microtype}
			\usepackage{lmodern}
			\graphicspath{{img/}}
			
			\renewcommand{\frontinstitutionfont}{\fontsize{14}{17}\bfseries\scshape}
			\renewcommand{\fronttitlefont}{\fontsize{17}{21}\bfseries\scshape}
			\renewcommand{\frontfootfont}{\fontsize{12}{14}\bfseries\scshape}
		\end{Preambolo*}
	\end{frontespizio}

%----------------------------------------------------------------------------------------
%	PAGINA BIANCA
%----------------------------------------------------------------------------------------
\newpage
\null
\thispagestyle{empty}
\newpage

%----------------------------------------------------------------------------------------
%	INTRO
%----------------------------------------------------------------------------------------
\chapter{Relazione attività di laboratorio 3}

\section{Circuito 1: raddrizzatore a doppia semionda di precisione}
\subsection{Schema del circuito e Funzione di Trasferimento}
Questo circuito, come si può notare dalla figura \ref{figura:schema1}, presenta due amplificatori operazionali, di cui quello in alto è retroazionato negativamente, e un diodo, il cui anodo è collegato all'uscita dell'OPAMP in basso e il cui catodo è collegato all'ingresso non invertente dell'OPAMP in alto.
\begin{figure}[h]
	\centering
	\includegraphics[height=7.5cm]{immagini/schema1}
	\caption{Schema del raddrizzatore a doppia semionda di precisione.}
	\label{figura:schema1}
\end{figure}
\\ \noindent La funzione di trasferimento di questo raddrizzatore è:
\begin{equation}
   \begin{cases}
   V_{in}< \SI{0}{\volt}\;\;\indent\indent\rightarrow \mathrm{D\;ON}\;\;\;\; \Rightarrow V_{out} = -V_{in}\\
   V_{in}\ge \SI{0}{\volt}\;\;\indent\indent\rightarrow \mathrm{D\; OFF}\;\; \Rightarrow V_{out} = V_{in}
   \end{cases}
\end{equation}
Da questa funzione di trasferimento si può notare che l'uscita non risulta shiftata rispetto all'ingresso, come invece succedeva nel raddrizzatore a doppia semionda analizzato durante lo scorso laboratorio. Dunque questo circuito permette di risolvere il problema della differenza di tensione presente tra uscita e ingresso. Di conseguenza, in uscita si ottiene un segnale analogo a quello in ingresso, ma in cui le semionde risultano raddrizzate.
\subsection{Analisi e dati sperimentali}
Per la realizzazione del circuito sulla breadboard (visibile nella figura \ref{figura:circuito1}) abbiamo deciso di utilizzare due amplificatori operazionali di tipo \textmu A741, che sono amplificatori operazionali \textit{general purpose}, e un diodo di tipo 1N4148.
Invece per quanto riguarda i valori delle resistenze abbiamo utilizzato resistenze da \SI{12}{k\ohm}, le cui misure sono state riportate nella tabella \ref{table:mis_res1}.
\begin{table}[h!]
	\centering
	\begin{tabular}{|c|c|c|}
		\cline{2-3} 
		\multicolumn{1}{c|}{} & \textbf{Valore nominale} & \textbf{Valore misurato}\\ 
		\hline
		$\mathbf{R_1}$ & \SI{12}{k\ohm} & \SI{11.802}{k\ohm} \\ 
		\hline
		$\mathbf{R_2}$ & \SI{12}{k\ohm} & \SI{11.947}{k\ohm} \\ 
		\hline
		$\mathbf{R_3}$ & \SI{12}{k\ohm} & \SI{11.885}{k\ohm} \\ 
		\hline
	\end{tabular}
	\caption{Misure delle resistenze utilizzate per il circuito.}
	\label{table:mis_res1}
\end{table}
\begin{figure}[h]
	\centering
	\includegraphics[height=6.5cm]{immagini/circuito1}
	\caption{Fotografia del raddrizzatore a doppia semionda di precisione realizzato in laboratorio.}
	\label{figura:circuito1}
\end{figure}
\\\indent Dopo aver realizzato il circuito sulla breadboard, sono state collegate sia le alimentazioni (con un valore di \SI{+10}{\volt} per l'alimentazione positiva e di \SI{-10}{\volt} per quella negativa) sia il segnale d'ingresso (con un'ampiezza picco-picco di \SI{2}{\volt} e frequenza prima di \SI{100}{\hertz} e poi pari a \SI{1}{k\hertz}).\par
Nella figura \ref{figura:uscita11} si può vedere il segnale in uscita prodotto dall'oscilloscopio. Per queste due frequenze sono state analizzate anche le rappresentazioni XY, ovvero la caratteristica tra tensione di ingresso e tensione di uscita del circuito (in figura \ref{figura:xyuscita1}).
\begin{figure}[h!]
	\centering
	\includegraphics[height=4.6cm]{immagini/TEK00000}
	\includegraphics[height=4.6cm]{immagini/TEK00003}
	\caption{Risposta del circuito con $\mathrm{f= \SI{100}{\hertz}}$ (sinistra) e con $\mathrm{f= \SI{1}{k\hertz}}$ (destra).}
	\label{figura:uscita11}
\end{figure}
\begin{figure}[h!]
	\centering
	\includegraphics[height=4.6cm]{immagini/TEK00001}
	\includegraphics[height=4.6cm]{immagini/TEK00010}
	\caption{Rappresentazione XY della risposta del circuito con $\mathrm{f= \SI{100}{\hertz}}$ (sinistra) e con $\mathrm{f= \SI{1}{k\hertz}}$ (destra).}
	\label{figura:xyuscita1}
\end{figure}
\begin{figure}[h!]
	\centering
	\includegraphics[height=4.6cm]{immagini/TEK00002}
	\includegraphics[height=4.6cm]{immagini/TEK00004}
	\caption{Risposta dell'OPAMP in basso con $\mathrm{f= \SI{100}{\hertz}}$ (sinistra) e con $\mathrm{f= \SI{1}{k\hertz}}$ (destra).}
	\label{figura:uscita111}
\end{figure}
\\In tutti questi grafici il segnale presenta un andamento quasi ideale, ma nella parte iniziale delle semionde negative raddrizzate si possono notare dei tratti anomali vicino all'asse delle ascisse. Nel grafico XY, li vediamo perché il segnale non si trova solo nel primo e nel secondo quadrante, ma c'è un tratto (seppur molto breve) anche nel terzo quadrante. Queste anomalie sono dovute al fatto che, quando il segnale in ingresso è positivo, il diodo è spento e quindi l'OPAMP in basso lavora in anello aperto. Dunque la sua tensione in uscita è pari a $\displaystyle\mathrm{V_{SS}}$, come si vede in figura \ref{figura:uscita111}, e quindi deve svolgere uno swing abbastanza ampio per raggiungere il valore dell'alimentazione. Questo provoca un aumento del tempo necessario per erogare una tensione positiva e di conseguenza il diodo non si accende istantaneamente. Inoltre in questo modo si ha un effetto anche sulla velocità del sistema nel raddrizzare il primo tratto delle semionde negative.
\\Analizzando invece frequenze maggiori (ad esempio \SI{10}{k\hertz}), il segnale presenta delle anomalie di maggiore rilevanza rispetto alle precedenti, come si può notare nella figura \ref{figura:uscita12}. Le semionde positive sono poco distorte, infatti vengono solo leggermente ritardate, invece sulle semionde negative si vede chiaramente l'escursione di tensione provocata dall'operazionale che si trova in anello aperto. Queste anomalie derivano dal fatto che l'OPAMP non è adatto a operare in alta frequenza quando presenta un anello che può risultare aperto.
\begin{figure}[h!]
	\centering
	\includegraphics[height=6.5cm]{immagini/TEK00008}
	\caption{Risposta del circuito con $\mathrm{f= \SI{10}{k\hertz}}$.}
	\label{figura:uscita12}
\end{figure}
\newpage
\section{Circuito 2: trigger di Schmitt}
\subsection{Schema del circuito e Funzione di Trasferimento}
In questo circuito è presente un amplificatore operazionale non retroazionato negativamente, ma con una retroazione positiva, quindi il circuito opera come un comparatore.\par
La rete di reazione positiva è formata da un partitore resistivo, in cui utilizzeremo due resistenze dello stesso valore per analizzare la risposta del circuito in maniera più semplice, come si può vedere dallo schema di figura \ref{figura:schema2}.\par
La particolarità del trigger di Schmitt è che, a differenza dei normali comparatori, la caratteristica ingresso-uscita presenta un'isteresi, di conseguenza questo circuito è immune a eventuali disturbi presenti in ingresso. In un comparatore semplice infatti, se il segnale in ingresso è rumoroso su valori prossimi alla soglia ${V^+}$, si possono avere delle transizioni involontarie dell'uscita da ${V_{DD}}$ a ${V_{SS}}$, perciò il rumore in ingresso rende anche l'uscita rumorosa. Questo problema viene risolto nel trigger di Schmitt, perché una volta attraversata una soglia (per esempio ${V_H^+}$ ) quest'ultima varierà il suo valore e si sposterà sul secondo valore (${V_L^+}$), quindi l'uscita non cambia finché non si attraversa la seconda soglia, rendendo il circuito robusto rispetto ai rumori presenti sulle linee d'ingresso.\par
\begin{figure}[h]
	\centering
	\includegraphics[height=7.5cm]{immagini/schema2}
	\caption{Schema del trigger di Schmitt.}
	\label{figura:schema2}
\end{figure}
La funzione di trasferimento di questo comparatore è:\\
\begin{equation}
   \begin{cases} % da controllare
   V_{out}= V_{DD}\;\indent\indent \mathrm{per\;}  V_L^+<V_{in}<V_H^+\mathrm{, \;\; con\;}\displaystyle{ V_H^+=\frac{R_1}{R_1+R_2}\cdot V_{DD}=\frac{V_{DD}}{2}}\mathrm{\;\;\;\; se \;} R_1=R_2\\[10pt]
   V_{out}= V_{SS}\;\;\indent\indent \mathrm{per\;} V_H^+<V_{in}<V_L^+\mathrm{, \;\; con\;}\displaystyle{ V_H^+=\frac{R_1}{R_1+R_2}\cdot |V_{SS}|=\frac{|V_{SS}|}{2}}\mathrm{\;\; se \;} R_1=R_2\\
   \end{cases}
\end{equation}
% SISTEMARE FDT
%\begin{equation}
%	\begin{cases} % da controllare
%		V_{out}= V_{DD}\;\indent\indent \mathrm{per\; t\; da\;} V_{in}=V_L^+\mathrm{\; a\;}V_{in}=V_H^+\mathrm{, \;\; con\;}\displaystyle{ V_H^+=\frac{R_1}{R_1+R_2}\cdot V_{DD}=\frac{V_{DD}}{2}}\mathrm{\;\;\;\; se \;} R_1=R_2\\[10pt]
%		V_{out}= V_{SS}\;\;\indent\indent \mathrm{per\; t\; da\;} V_{in}=V_H^+\mathrm{\; a\;}V_{in}=V_L^+\mathrm{, \;\; con\;}\displaystyle{ V_H^+=\frac{R_1}{R_1+R_2}\cdot |V_{SS}|=\frac{|V_{SS}|}{2}}\mathrm{\;\; se \;} R_1=R_2\\
%	\end{cases}
%\end{equation}
\subsection{Analisi e dati sperimentali}
Per la realizzazione del circuito sulla breadboard, che viene mostrata nella figura \ref{figura:circuito2}, è stato utilizzato un amplificatore operazionale di tipo \textmu A741. Invece per quanto riguarda i valori delle resistenze abbiamo utilizzato due resistenze da \SI{12}{k\ohm}, le cui misure sono state riportate nella tabella \ref{table:mis_res2}.
\begin{table}[h!]
	\centering
	\begin{tabular}{|c|c|c|}
		\cline{2-3} 
		\multicolumn{1}{c|}{} & \textbf{Valore nominale} & \textbf{Valore misurato}\\ 
		\hline
		$\mathbf{R_1}$ & \SI{12}{k\ohm} & \SI{11.802}{k\ohm} \\ 
		\hline
		$\mathbf{R_2}$ & \SI{12}{k\ohm} & \SI{11.947}{k\ohm} \\ 
		\hline
	\end{tabular}
	\caption{Misure delle resistenze utilizzate per il circuito.}
	\label{table:mis_res2}
\end{table}
\begin{figure}[h]
	\centering
	\includegraphics[height=7.5cm]{immagini/circuito2}
	\caption{Fotografia del trigger di Schmitt realizzato in laboratorio.}
	\label{figura:circuito2}
\end{figure}
\\Dopo aver realizzato il circuito sulla breadboard, scegliamo di applicare in ingresso un'onda triangolare con ampiezza picco-picco di \SI{15}{\volt}. Questo valore è stato scelto in modo tale da poter vedere le transizioni da una soglia all'altra.\par
Per prima cosa abbiamo analizzato l'uscita del circuito con un segnale in ingresso con frequenza di \SI{100}{\hertz} (figura \ref{figura:uscita21}), dai grafici si può notare che il comportamento dei segnali risulta corretto. Abbiamo quindi misurate le due soglie, i cui valori sono risultati pari a \SI{4.88}{\volt} per ${V_H^+}$ (soglia positiva) e \SI{-4.16}{\volt} per ${V_L^+}$ (soglia negativa). Anche questi valori sono corretti, infatti corrispondono a circa metà della tensione che l'OPAMP eroga in uscita quando satura al valore positivo o negativo.
\begin{figure}[h!]
	\centering
	\includegraphics[height=4.6cm]{immagini/TEK00017}
	\includegraphics[height=4.6cm]{immagini/TEK00018}
	\caption{Risposta del circuito con $\mathrm{f= \SI{100}{\hertz}}$.}
	\label{figura:uscita21}
\end{figure}
\\\indent Successivamente è stato analizzato anche il grafico della caratteristica ingresso-uscita per la stessa frequenza. Come si può notare dalla figura \ref{figura:uscita22}, è presente un ciclo di isteresi, come ci si aspettava per questo trigger. L'area del ciclo di isteresi dipende dal valore delle due resistenze secondo la formula:
\\[2pt]\indent$\displaystyle{\mathrm{Isteresi}=2\cdot\frac{R_1}{R_1+R_2}\cdot |V_ {DD}|}$
\begin{figure}[h!]
	\centering
	\includegraphics[height=6cm]{immagini/TEK00019}
	\caption{Rappresentazione XY della risposta del circuito con $\mathrm{f= \SI{100}{\hertz}}$.}
	\label{figura:uscita22}
\end{figure}
\\In seguito abbiamo analizzato frequenze maggiori, ad esempio \SI{1}{k\hertz}, e abbiamo notato che dalla figura \ref{figura:uscita23} risulta visibile lo slew rate dell'OPAMP, ovvero il tempo di cui necessita per far passare la sua uscita dall'alimentazione positiva a quella negativa e viceversa.
\\La presenza dello slew rate causa nella figura \ref{figura:uscita24} una deformazione del ciclo di isteresi, che non risulta più con una forma rettangolare ma presenta la forma di un parallelogramma.
\begin{figure}[h!]
	\centering
	\includegraphics[height=6cm]{immagini/TEK00020}
	\caption{Risposta del circuito con $\mathrm{f= \SI{1}{k\hertz}}$.}
	\label{figura:uscita23}
\end{figure}
\begin{figure}[h!]
	\centering
	\includegraphics[height=6cm]{immagini/TEK00022}
	\caption{Rappresentazione XY della risposta del circuito con $\mathrm{f= \SI{1}{k\hertz}}$.}
	\label{figura:uscita24}
\end{figure}

\newpage\null\newpage % da togliere poi ultimi due comandi
\section{Circuito 3: oscillatore con duty cicle pari a 50\%}
\subsection{Schema del circuito e Funzione di Trasferimento}
Il trigger di Schmitt può essere utilizzato per realizzare un oscillatore. Un oscillatore è un circuito  che produce in uscita una forma d'onda ad una determinata frequenza, entrambi i parametri (forma e frequenza) dipendono sia dai componenti presenti nell'oscillatore che dalle tensioni di alimentazione. \`E un circuito particolare perché è \textit{autoalimentato}, questo significa che non dovremo applicare un segnale in ingresso per osservare il segnale d'uscita (per innescare la reazione è sufficiente un disturbo, ad esempio il rumore dei componenti elettronici).
\begin{figure}[h!]
	\centering
	\includegraphics[height=8cm]{immagini/schema3}
	\caption{Schema dell'oscillatore con duty cicle di 50\%.}
	\label{figura:schema3}
\end{figure}
\\Il circuito che realizzeremo produrrà in uscita un'onda quadra. Inizialmente utilizzeremo lo stesso valore per le resistenze $\mathrm{R_1}$ e $\mathrm{R_2}$, quindi le soglie che determinano le transizioni da alto a basso e viceversa sono:
\begin{table}[h!]
	\centering
	\begin{tabular}{cc}
		$\displaystyle{V_H^+=\frac{R_1}{R_1+R_2}\cdot V_{DD}=+\SI{5}{\volt}}$\;\;\;\;\;\;\;\;\;\;\;\;\;\;\;\;\;\;\;\; & $\displaystyle{V_L^+=\frac{R_1}{R_1+R_2}\cdot V_{SS}=-\SI{5}{\volt}}$\\ 
	\end{tabular}
\end{table}
\\Il suo funzionamento sfrutta il processo di carica e scarica di un condensatore su una resistenza. Ipotizziamo che inizialmente il condensatore sia scarico, se $\mathrm{V^-}$ è minore di $\mathrm{V^+}$, l'uscita satura alla tensione positiva e il condensatore si carica attraverso la resistenza $\mathrm{R}$. La tensione ai suoi capi cresce con andamento esponenziale, con costante di tempo $\displaystyle{\tau=R\cdot C}$. Se non ci fosse la soglia $\mathrm{V_H^+}$, la tensione ai capi del condensatore continuerebbe a crescere fino a raggiungere la tensione di saturazione positiva. In questo oscillatore invece, quando la tensione ai capi della capacità raggiunge la soglia $\mathrm{V_H^+}$, si ha che $\mathrm{V^-}$ è maggiore di $\mathrm{V^+}$, di conseguenza l'uscita satura alla tensione negativa. Questo provoca la scarica della capacità sulla resistenza, l'andamento è sempre esponenziale con la stessa costante di tempo $\tau$. Il processo di scarica continua fin quando la tensione ai capi della capacità risulta pari alla soglia inferiore, in questo caso risulta soddisfatta la condizione $V^+>V^-$, e il condesatore ripete il processo di carica. Riassumendo, il circuito si comporta come un comparatore con isteresi.\par
Il periodo di oscillazione è dato dalla somma dell'intervallo di tempo in cui l'onda quadra resta alta ($T_1$) e dell'intervallo di tempo in cui l'onda quadra resta bassa ($T_2$). Questi valori si ricavano dalle formule inverse di carica e di scarica del condensatore:
\begin{table}[h!]
	\centering
	\begin{tabular}{cc}
		$\displaystyle{T_1=\tau\cdot\ln\frac{V_L^+-V_{DD}}{V_H^+-V_{DD}}=\tau\cdot\ln\frac{R_2+2R_1}{R_2}}$\;\;\;\;\;\;\;\;\;\;\;\;\; & $\displaystyle{T_2=\tau\cdot\ln\frac{V_H^+-V_{SS}}{V_L^+-V_{SS}}=\tau\cdot\ln\frac{R_2+2R_1}{R_2}}$\\ 
	\end{tabular}
	\label{table:formuleSoglie}
\end{table}
\\Dato che il circuito è simmetrico, le due soglie sono in modulo uguali. Inoltre, dato che le due resistenze hanno lo stesso valore, l'argomento del logaritmo è uguale per entrambi gli intervalli, di conseguenza $T_1=T_2$. Il duty cycle, $\delta$, che è la percentuale di tempo in cui un'onda quadra è alta, sarà pari a:
\\[4pt]\indent$\displaystyle{\delta=\frac{T_1}{T}=\frac{T_1}{T_1+T_2}=\frac{T_1}{2\cdot T_1}=\frac{1}{2}=50\%}$
\subsection{Analisi e dati sperimentali}
Utilizziamo ancora le tre resistenza dal valore nominale di \SI{12}{k\ohm} e di cui abbiamo già riportato le misure in tabella \ref{table:mis_res1}. Posizioniamo i componenti sulla breadboard, come mostrato in figura \ref{figura:circuito3}, e colleghiamo l'oscilloscopio. I nodi di cui siamo interessati a misurare la tensione sono gli ingressi dell'OPAMP (sia quello invertente che quello non invertente) e l'uscita dell'operazionale.
\begin{figure}[h!]
	\centering
	\includegraphics[height=7cm]{immagini/circuito3}
	\caption{Fotografia dell'oscillatore con duty cicle pari a 50\% realizzato in laboratorio.}
	\label{figura:circuito3}
\end{figure}
\\In figura \ref{figura:oscillo1} è riportato il grafico della tensione ai capi della capacità (traccia gialla, CH1) e della tensione in uscita all'OPAMP (traccia azzurra, CH2). Tramite l'oscilloscopio ricaviamo che la frequenza dell'onda quadra è pari a \SI{247}{\hertz}. Successivamente, confrontiamo questi valori con quelli teorici.
\begin{figure}[h!]
	\centering
	\includegraphics[height=7cm]{immagini/TEK00025}
	\caption{Grafico della tensione ai capi della capacità e della tensione in uscita all'oscillatore.}
	\label{figura:oscillo1}
\end{figure}
\\Utilizziamo le formule ricavate nella sezione precedente per calcolare i due intervalli di tempo:
\\[4pt]\indent$\displaystyle{T_1=T_2=\tau\cdot\ln\frac{R_2+2R_1}{R_2}=\SI{12}{k\ohm}\cdot\SI{150}{n\farad}\cdot\ln\frac{\SI{12}{k\ohm}+2\cdot\SI{12}{k\ohm}}{\SI{12}{k\ohm}}=\SI{1.978}{m\second}}$
\\[4pt]\indent$\displaystyle{T=2\cdot T_1 = \SI{3.955}{m\second}}$
\\[4pt]\indent$\Rightarrow\displaystyle{f=\frac{1}{T}=\frac{1}{\SI{3.955}{m\second}}=\SI{253}{\hertz}}$
\\[4pt]Il circuito si comporta correttamente, come vediamo dal grafico. Il valore di frequenza che misuriamo è corretto e preciso, infatti è maggiore del 2.4\% rispetto al valore teorico.\par
Dal grafico in figura \ref{figura:soglieCap}, possiamo vedere che il ciclo di carica e scarica della capacità è condizionato dalle due soglie, proprio come ci aspettiamo dalla teoria.
\begin{figure}[h!]
	\centering
	\includegraphics[height=7cm]{immagini/TEK00026}
	\caption{Tensione ai capi della capacità rispetto alle soglie del circuito.}
	\label{figura:soglieCap}
\end{figure}
\\Se volessimo un'onda quadra con duty cycle diverso da 50\%, possiamo variare il valore delle alimentazioni, dato che i due intervalli di tempo $\mathrm{T_1}$ e $\mathrm{T_2}$ (e di conseguenza anche il duty cyle) dipendono da queste. Lasciamo invariata la tensione di alimentazione positiva, invece quella negativa la impostiamo a \SI{-15}{\volt}. Misuriamo ancora la tensione ai capi della capacità e la tensione in uscita, il grafico è mostrato in figura \ref{figura:oscillo3_2}.
\begin{figure}[h!]
	\centering
	\includegraphics[height=7cm]{immagini/TEK00032}
	\caption{Tensione ai capi della capacità ed in uscita all'oscillatore con le alimentazioni cambiate.}
	\label{figura:oscillo3_2}
\end{figure}
\\Dato che la soglia negativa è più bassa, ci aspettiamo un duty cycle superiore a 50\%, perché l'onda quadra rimarrà bassa per un intervallo di tempo minore rispetto al caso precedente. Dal grafico in figura \ref{figura:oscillo3_2} vediamo che quest'affermazione è corretta, infatti il valore di duty cycle misurato dall'oscilloscopio è di 55.20\%. Il periodo dell'onda quadra misurato dall'oscilloscopio è di \SI{4.090}{m\second}, verifichiamo se questo valore è corretto con le formule per il calcolo di $T_1$ e $T_2$:
\begin{table}[h!]
	\begin{tabular}{cc}
		\;\;\;$\displaystyle{V_H^+=\frac{R_1}{R_1+R_2}\cdot V_{DD}=\frac{1}{2}\cdot V_{DD}=+\SI{5}{\volt}}$\;\;\;\;\;\;\;\;\;\;\;\;\;\;\;\;\;\;\;\;\;\;\; & $\displaystyle{V_L^+=\frac{R_1}{R_1+R_2}\cdot V_{SS}=\frac{1}{2}\cdot V_{SS}=-\SI{7.5}{\volt}}$\\ 
	\end{tabular}
\end{table}
\\\indent$\rightarrow\displaystyle{T_1=\tau\cdot\ln\frac{V_L^+-V_{DD}}{V_H^+-V_{DD}}=R\cdot C\cdot\ln\frac{-7.5-10}{5-10}=\SI{12}{k\ohm}\cdot\SI{150}{n\farad}\cdot\ln\frac{7}{2}}=\SI{2.255}{m\second}$
\\[6pt]\indent$\rightarrow\displaystyle{T_2=\tau\cdot\ln\frac{V_H^+-V_{SS}}{V_L^+-V_{SS}}=R\cdot C\cdot\ln\frac{5+15}{-7.5+15}=\SI{12}{k\ohm}\cdot\SI{150}{n\farad}\cdot\ln\frac{8}{3}}=\SI{1.765}{m\second}$
\\[6pt]\indent$\Rightarrow\displaystyle{T=T_1+T_2=\SI{2.255}{m\second}+\SI{1.765}{m\second}=\SI{4.020}{m\second}}$
\\[2pt]Il risultato precedente è corretto, dato che è inferiore solo del 1.7\% rispetto al valore misurato dall'oscilloscopio.
\newpage
\section{Circuito 4: oscillatore con duty cicle diverso da 50\%}
\subsection{Schema del circuito e Funzione di Trasferimento}
Spostare le tensioni di alimentazione per far variare il duty cycle non è l'approccio migliore che si può adottare. Il circuito precedente può essere modificato per fare in modo che il condensatore non si carica e scarica sulla stessa resistenza, ma su due resistenze diverse. Per fare questo, complichiamo leggermente il circuito e al posto della resistenza $\mathrm{R}$ sostituiamo due resistenze di diverso valore, $\mathrm{R_3}$ e $\mathrm{R_4}$. Successivamente, facciamo in modo che i percorsi di carica e scarica della capacità siano differenti, perciò inseriamo due diodi: il primo diodo, $\mathrm{D_1}$, ha l'anodo collegato alla resistenza, mentre il catodo è collegato all'uscita dell'OPAMP, quindi la resistenza $\mathrm{R_3}$ sarà quella utilizzata per scaricare il condensatore; l'altro diodo invece, $\mathrm{D_2}$, è collegato in modo opposto rispetto al diodo $\mathrm{D_1}$, perciò la capacità si caricherà sulla resistenza $\mathrm{R_4}$. Il circuito risultante è mostrato in figura \ref{figura:schema4}.
\begin{figure}[h!]
	\centering
	\includegraphics[height=9cm]{immagini/schema4}
	\caption{Schema dell'oscillatore con duty cicle diverso da 50\%.}
	\label{figura:schema4}
\end{figure}
\\Scegliendo opportunamente i valori delle due resistenze $\mathrm{R_3}$ e $\mathrm{R_4}$, è possibile ottenere tempi di carica e scarica diversi, senza andare a modificare le tensioni di alimentazione o il valore delle soglie. In particolare, se $\mathrm{R_3}$ è minore di $\mathrm{R_4}$, il duty cycle sarà maggiore del 50\% perché l'onda quadra resterà alta per un tempo maggiore del semiperiodo, viceversa sarà minore.  Il funzionamento del circuito però non cambia, resterà sempre un oscillatore. 
\subsection{Analisi e dati sperimentali}
Scegliamo di utilizzare lo stesso valore del circuito precedente per le resistenze $\mathrm{R_1}$ e $\mathrm{R_2}$, anche la resistenza $\mathrm{R_3}$ la scegliamo da \SI{12}{k\ohm}. La resistenza $\mathrm{R_4}$ invece ha valore nominale di \SI{22}{k\ohm} e valore misurato di \SI{23.945}{k\ohm}. La fotografia di questo circuito è mostrata in figura \ref{figura:circuito4_1}.\par
Le formule utilizzate per il circuito precedente restano valide anche per questo circuito, cambiano solo le costanti di tempo $\mathrm{\tau}$ perché sono diverse per la fase di carica e di scarica. In particolare, dato che $\displaystyle{R_4>R_3}$, si ha che $\displaystyle{\tau_4>\tau_3}$, perciò la fase di carica sarà più lenta di quella di scarica, quindi l'intervallo di tempo in cui l'onda quadra è alta è maggiore rispetto a quello in cui è bassa e di conseguenza il duty cycle sarà maggiore del 50\%.
\begin{figure}[h!]
	\centering
	\includegraphics[height=7cm]{immagini/circuito4_1.jpg}
	\caption{Fotografia dell'oscillatore con duty cicle maggiore del 50\% realizzato in laboratorio.}
	\label{figura:circuito4_1}
\end{figure}
\\Colleghiamo l'oscilloscopio per vedere l'andamento della tensione ai capi del condensatore e della tensione in uscita al circuito. Sempre con l'oscilloscopio, misuriamo il periodo dell'onda quadra prodotta ed il suo duty cycle. Il grafico è riportato in figura \ref{figura:oscillo4_1}.\par
\begin{figure}[h!]
	\centering
	\includegraphics[height=6.9cm]{immagini/TEK00033}
	\caption{Tensione ai capi della capacità ed in uscita all'oscillatore con $\delta>50\%$.}
	\label{figura:oscillo4_1}
\end{figure}
Il duty cycle misurato dall'oscilloscopio è pari a 53.07\%, perciò è maggiore del 50\% come ci aspettavamo. Vediamo infatti che la fase di carica della capacità è molto più lenta della fase di scarica, perché la prima fase ha una durata temporale maggiore della seconda. \par
Il periodo misurato è di \SI{6.497}{m\second}, pari ad una frequenza di \SI{154}{\hertz}. Verifichiamo che le misure corrispondano a quanto ricaviamo dalla teoria:
\begin{table}[h!]
	\begin{tabular}{cc}
		\;\;\;$\displaystyle{V_H^+=\frac{R_1}{R_1+R_2}\cdot V_{DD}=+\SI{5}{\volt}}$\;\;\;\;\;\;\;\;\;\;\;\;\;\;\;\;\;\;\;\;\;\;\; & $\displaystyle{V_L^+=\frac{R_1}{R_1+R_2}\cdot V_{SS}=-\SI{5}{\volt}}$\\ 
	\end{tabular}
\end{table}
\\\indent$\rightarrow\displaystyle{T_1=\tau_1\cdot\ln\frac{V_L^+-V_{DD}}{V_H^+-V_{DD}}=R_4\cdot C\cdot\ln\frac{-5-10}{5-10}=\SI{23.945}{k\ohm}\cdot\SI{150}{n\farad}\cdot\ln3}=\SI{3.946}{m\second}$
\\[6pt]\indent$\rightarrow\displaystyle{T_2=\tau_2\cdot\ln\frac{V_H^+-V_{SS}}{V_L^+-V_{SS}}=R_3\cdot C\cdot\ln\frac{5+10}{-5+10}=\SI{11.947}{k\ohm}\cdot\SI{150}{n\farad}\cdot\ln3}=\SI{1.969}{m\second}$
\\[6pt]\indent$\Rightarrow\displaystyle{T=T_1+T_2=\SI{3.946}{m\second}+\SI{1.969}{m\second}=\SI{5.915}{m\second}}$
\\[6pt]\indent$\Rightarrow\displaystyle{\delta=\frac{T_1}{T}=\frac{\SI{3.946}{m\second}}{\SI{5.915}{m\second}}= 66.71\%}$
\\[4pt]Otteniamo un periodo inferiore di circa \SI{0.5}{m\second} ed un duty cycle sensibilmente maggiore rispetto alla misura dell'oscilloscopio. Queste anomalie potrebbero essere causate dal rumore, infatti vediamo che sia l'onda quadra che la tensione ai capi della capacità sono segnali piuttosto rumorosi, questo può provocare l'imprecisione delle misure dell'oscilloscopio.\par
Nel grafico seguente, la figura \ref{figura:oscillo4_1_soglie}, vediamo invece l'andamento della tensione ai capi della capacità rispetto alle soglie. Notiamo che in questo caso la misura del duty cycle è più vicina al valore calcolato precedentemente con le formule.
\begin{figure}[h!]
	\centering
	\includegraphics[height=6.9cm]{immagini/TEK00038}
	\caption{Tensione ai capi della capacità rispetto alle soglie dell'oscillatore con $\delta>50\%$.}
	\label{figura:oscillo4_1_soglie}
\end{figure}
\\Infine, proviamo ad invertire l'orientamento dei due diodi. Così facendo, andiamo ad invertire il ruolo delle due resistenze e di conseguenza andiamo a scambiare fra di loro i percorsi di carica e scarica della capacità. Questa nuova versione è mostrata in figura \ref{figura:circuito4_2}.
\begin{figure}[h!]
	\centering
	\includegraphics[height=7cm]{immagini/circuito4_2.jpg}
	\caption{Fotografia dell'oscillatore con duty cicle minore del 50\% realizzato in laboratorio.}
	\label{figura:circuito4_2}
\end{figure}
\\Ripetiamo l'analisi e le misure fatte per il circuito precedente. In questo caso ci aspettiamo un duty cycle minore del 50\%, dato che la fase di scarica della capacità ha una durata superiore rispetto alla fase di carica. L'andamento della tensione ai capi del condensatore e della tensione in uscita al circuito, insieme alle misure del periodo e del duty cycle, sono riportate in figura \ref{figura:oscillo4_2}.
\begin{figure}[h!]
	\centering
	\includegraphics[height=6.9cm]{immagini/TEK00034}
	\caption{Tensione ai capi della capacità ed in uscita all'oscillatore con $\delta<50\%$.}
	\label{figura:oscillo4_2}
\end{figure}
\\I valori delle due soglie restano invariati ($V_H^+=+\SI{5}{\volt}$, $V_L^+=\SI{-5}{\volt}$). Calcoliamo i parametri di nostro interesse e vedremo che i valori di $\mathrm{T_1}$ e di $\mathrm{T_2}$ si scambiano tra di loro, perciò il periodo, e di conseguenza anche la frequenza, resteranno invariati rispetto al circuito precedente. 
\\[4pt]\indent$\rightarrow\displaystyle{T_1=\tau_2\cdot\ln\frac{V_L^+-V_{DD}}{V_H^+-V_{DD}}=R_3\cdot C\cdot\ln\frac{-5-10}{5-10}=\SI{11.947}{k\ohm}\cdot\SI{150}{n\farad}\cdot\ln3}=\SI{1.969}{m\second}$
\\[6pt]\indent$\rightarrow\displaystyle{T_2=\tau_1\cdot\ln\frac{V_H^+-V_{SS}}{V_L^+-V_{SS}}=R_4\cdot C\cdot\ln\frac{5+10}{-5+10}=\SI{23.945}{k\ohm}\cdot\SI{150}{n\farad}\cdot\ln3}=\SI{3.946}{m\second}$
\\[6pt]\indent$\Rightarrow\displaystyle{T=T_1+T_2=\SI{3.946}{m\second}+\SI{1.969}{m\second}=\SI{5.915}{m\second}}$
\\[6pt]\indent$\Rightarrow\displaystyle{\delta=\frac{T_1}{T}=\frac{\SI{1.969}{m\second}}{\SI{5.915}{m\second}}= 33.29\%}$
\\[4pt]Anche in questo caso i valori teorici e reali sono sensibilmente diversi, però i segnali prodotti sono sempre piuttosto rumorosi. Infatti, il duty cycle che calcola l'oscilloscopio non è molto preciso, ma se guardiamo il grafico di figura \ref{figura:oscillo4_2_soglie}, in cui viene mostrato l'andamento della tensione ai capi della capacità rispetto alle soglie, notiamo che il valore è molto più preciso, esattamente come è avvenuto per il circuito precedente. La misura del periodo rimane sempre parecchio imprecisa.
\begin{figure}[h!]
	\centering
	\includegraphics[height=6.9cm]{immagini/TEK00036}
	\caption{Tensione ai capi della capacità rispetto alle soglie dell'oscillatore con $\delta<50\%$.}
	\label{figura:oscillo4_2_soglie}
\end{figure}

%----------------------------------------------------------------------------------------

\end{document}
