
\documentclass{report}

\usepackage[utf8]{inputenc}
\usepackage[italian]{babel}
\usepackage{import}
\usepackage{todonotes}
\usepackage{color}
\usepackage{rotating}
\usepackage[hidelinks]{hyperref}
\usepackage{url}
\usepackage{pdfpages}
\usepackage{siunitx}
\usepackage{pdflscape}
\usepackage{subfig}
\usepackage[euler]{textgreek}
\usepackage{mhchem}

\usepackage{multirow}

\usepackage{enumerate} 
\usepackage{amsmath}
\usepackage{amsfonts}

\usepackage[signatures,swapnames,sans]{frontespizio}

\usepackage{geometry}
\geometry{portrait, margin=3cm}
\usepackage{siunitx}
\usepackage{booktabs}

\renewcommand*\figurename{Figura}

\newcommand{\sub}[1]{\textsubscript{#1}}
\newcommand{\super}[1]{\textsuperscript{#1}}
\newcommand{\parallelsum}{\mathbin{\!/\mkern-5mu/\!}}

\newcommand{\Fig}[0]{Fig.}

\usepackage{titlesec}

\titleformat{\chapter}{\normalfont\huge}{}{20pt}{\huge\bfseries}

\linespread{1.3}


%% COMANDI UTILI
%\begin{table}[h]
%	\centering
%	\begin{tabular}{|c|c|c|}
%	\cline{2-3} 
%	\multicolumn{1}{c|}{} & \textbf{Valore nominale} & \textbf{Valore misurato}\\ 
%		%\hline
%		%{} & \textbf{Valore nominale} & \textbf{Valore misurato} \\ 
%		\hline
%		$\mathbf{R_1}$ & \SI{18}{k\ohm} & \SI{17.977}{k\ohm} \\ 
%		\hline
%		$\mathbf{R_2}$& \SI{1.8}{k\ohm} & \SI{1.815}{k\ohm} \\ 
%		\hline
%	\end{tabular}
%\caption{Misure delle resistenze utilizzate per il circuito.}
%\label{table:mis_res}
%\end{table}
%\begin{figure}[h!]
%\centering
%\includegraphics[height=6.5cm]{immagini/TEK00018}\\(a)\\[1ex]
%\includegraphics[height=6.5cm]{immagini/TEK00019}\\(b)
%\caption{Risposta del circuito con accoppiamento DC (a) e accoppiamento AC (b).}
%	\label{figura:accopp}
%\end{figure}

\begin{document}
\addtocounter{chapter}{+3}
	\begin{frontespizio}
		\Margini{3cm}{3cm}{3cm}{3cm}
		\Universita{Bergamo}
		\Logo[43.332mm]{unibg-mark}
		\Divisione{Scuola di Ingegneria}
		\Corso[Laurea Magistrale]{Ingegneria Informatica}
		\Titolo{Laboratorio di Elettronica}
		\Sottotitolo{Relazione esperienza di laboratorio 4}
		\Punteggiatura{}
		\NRelatore{Prof.}{Prof.}
		\Relatore{Luigi Gaioni}
		\Candidato[1058231]{Giulia Allievi}
		\Candidato[1059640]{Martina Fanton}
		\Annoaccademico{2022--2023}
		\begin{Preambolo*}
			\usepackage[italian]{babel}
			\usepackage[T1]{fontenc}
			\usepackage[utf8]{inputenc}
			\usepackage{microtype}
			\usepackage{lmodern}
			\graphicspath{{img/}}
			
			\renewcommand{\frontinstitutionfont}{\fontsize{14}{17}\bfseries\scshape}
			\renewcommand{\fronttitlefont}{\fontsize{17}{21}\bfseries\scshape}
			\renewcommand{\frontfootfont}{\fontsize{12}{14}\bfseries\scshape}
		\end{Preambolo*}
	\end{frontespizio}

%----------------------------------------------------------------------------------------
%	PAGINA BIANCA
%----------------------------------------------------------------------------------------
\newpage
\null
\thispagestyle{empty}
\newpage

%----------------------------------------------------------------------------------------
%	INTRO
%----------------------------------------------------------------------------------------
\chapter{Relazione attività di laboratorio 4}
\section*{Introduzione}
Nei circuiti analizzati durante questo laboratorio sono presenti diodi, amplificatori operazionali e un nuovo circuito integrato, il timer 555.\par
Nella figura \ref{figura:timer1} a sinistra, si può vedere la numerazione e la denominazione di ciascun pin di questo componente. Per capire quali sono i terminali, sul package è presente una mezza luna in corrispondenza del pin numero 1 e poi la numerazione prosegue in senso antiorario. Invece nella stessa figura a destra si possono notare i componenti interni di questo circuito integrato e in particolare della tipologia di timer LM555, ovvero quella che abbiamo utilizzato durante questo laboratorio. Il nome del timer 555 è dovuto al fatto che nel suo schema interno sono presenti tre resistenze in serie da \SI{5}{k\ohm} l'una.
\begin{figure}[h!]
	\centering
	\includegraphics[height=4.6cm]{immagini/timer1}
	\includegraphics[height=4.6cm]{immagini/timer2}
	\caption{Package (a sinistra) e contenuto (a destra, fonte: \textcolor{blue}{\underline{\href{https://www.ti.com/lit/ds/symlink/lm555.pdf?ts=1667144089940&ref_url=https\%253A\%252F\%252Fwww.ti.com\%252Fproduct\%252FLM555}{datasheet}}} del LM555) del timer 555.}
	\label{figura:timer1}
\end{figure}
\\Questo componente richiede un'alimentazione singola per poter funzionare correttamente: la tensione positiva $\mathrm{V_{CC}}$ va collegata al pin 8, mentre la massa al pin 1.\par
Il timer 555 può essere utilizzato in tre modalità: astabile, monostabile e bistabile. In questo laboratorio abbiamo utilizzato la seconda configurazione, che è rappresentata nella figura \ref{figura:timer2}.
\begin{figure}[h!]
	\centering
	\includegraphics[height=4.6cm]{immagini/timer3}
	\caption{Configurazione monostabile del timer LM555 (fonte: \textcolor{blue}{\underline{\href{https://www.ti.com/lit/ds/symlink/lm555.pdf?ts=1667144089940&ref_url=https\%253A\%252F\%252Fwww.ti.com\%252Fproduct\%252FLM555}{datasheet}}} del LM555).}
	\label{figura:timer2}
\end{figure}

\newpage
\section{Circuito 1: circuito monostabile con trigger di Schmitt}
\subsection{Schema del circuito e Funzione di Trasferimento}
Questo circuito (in figura \ref{figura:schema1}) è stato ottenuto apportando delle modifiche all'oscillatore analizzato nel precedente laboratorio. In particolare sono stati aggiunti un diodo (con il catodo collegato a massa e con l'anodo collegato alla reazione negativa) e una rete di filtraggio (situata all'ingresso non invertente dell'amplificatore e collegata al circuito tramite un ulteriore diodo). Entrambi i diodi utilizzati sono di tipo 1N4148.\par
Per riuscire a visualizzare correttamente tutti i segnali sull'oscilloscopio, abbiamo sostituito l'amplificatore \textmu A741 utilizzato nei precedenti laboratori con un OPAMP di tipo TL071, perché abbiamo osservato sperimentalmente che era più performante per questo particolare circuito.
\begin{figure}[h!]
	\centering
	\includegraphics[height=7.5cm]{immagini/schema1}
	\caption{Schema del circuito monostabile con trigger di Schmitt.}
	\label{figura:schema1}
\end{figure}
\\\` E un circuito \textit{monostabile} (detto anche \textit{one-shot}) perché presenta uno stato stabile, ovvero quello in cui la tensione di uscita $\displaystyle\mathrm{V_{OUT}}$ si trova a un valore pari a $\displaystyle\mathrm{V_{DD}}$, ma può andare allo stato instabile in seguito ad uno stimolo esterno ed in uscita si avrà un impulso negativo fino a $\displaystyle\mathrm{V_{SS}}$ con una durata definita.  Questa durata dipende dal processo di carica del condensatore verso il valore di $\displaystyle\mathrm{V_{SS}}$ e questo processo viene interrotto quando viene raggiunta la soglia $\displaystyle\mathrm{V_L^+}$. Difatti la tensione sul condensatore può essere calcolata come:
\\[4pt]\indent$\displaystyle{\mathrm{V_C(t)}=\mathrm{V^-(t)}=\mathrm{V_{SS}}+(0.7-\mathrm{V_{SS}}) \cdot e^{-\frac{t-t_0}{\tau}}}$
\\[2pt]dove $\mathrm{t_0}$ è l'istante in cui si ha il fronte di discesa del segnale in ingresso $\mathrm{V_{IN}}$ e l'inizio del processo di carica del condensatore verso il valore $\mathrm{V_{SS}}$, mentre $\mathrm{\tau}$ corrisponde alla costante di tempo data da: \\\indent$\displaystyle{\mathrm{\tau}= R \cdot C}$.
\\Da questa equazione si può ricavare la formula della durata dell'impulso negativo in uscita, sostituendo:\\
$\indent t\rightarrow T_A$; \indent$t_0\rightarrow$ \SI{0}{\second}; \indent $\displaystyle{V^-(t)\rightarrow V_L^+=\frac{R_1}{R_1+R_2}\cdot V_{SS}}$;
\\[4pt]\indent\indent$\displaystyle{\Rightarrow T_A=\tau\cdot\ln\biggl(1+\frac{R_1}{R_2}\biggr)\indent \mathrm{se\;}|V_{SS}|\gg\SI{0.7}{\volt}}$
\\[4pt]Terminato quest'intervallo di tempo, il circuito ritornerà automaticamente dallo stato instabile al suo stato stabile. Il circuito rimane al suo stato stabile fin quando non sarà nuovamente perturbato dall'esterno. I circuiti monostabili sono utili per realizzare dei temporizzatori.
\subsection{Analisi e dati sperimentali}
Per poter costruire il circuito sulla breadboard, per prima cosa sono stati scelti i valori dei componenti da utilizzare. Nel caso delle resistenze, le loro misure sono state riportate nella tabella \ref{table:mis_res1}, mentre per i condensatori sono state utilizzate delle capacità di valore nominale di \SI{150}{n\farad} per C e di \SI{1}{n\farad} per $\mathrm{C_T}$. 
\begin{table}[h!]
	\centering
	\begin{tabular}{|c|c|c|}
		\cline{2-3} 
		\multicolumn{1}{c|}{} & \textbf{Valore nominale} & \textbf{Valore misurato}\\ 
		\hline
		$\mathbf{R}$ & \SI{12}{k\ohm} & \SI{11.882}{k\ohm} \\ 
		\hline
		$\mathbf{R_1}$ & \SI{12}{k\ohm} & \SI{11.934}{k\ohm} \\ 
		\hline
		$\mathbf{R_2}$ & \SI{12}{k\ohm} & \SI{11.950}{k\ohm} \\ 
		\hline
		$\mathbf{R_T}$ & \SI{12}{k\ohm} & \SI{11.894}{k\ohm} \\ 
		\hline
	\end{tabular}
	\caption{Misure delle resistenze utilizzate per il circuito.}
	\label{table:mis_res1}
\end{table}
\\Una volta costruito il circuito (figura \ref{figura:circuito1}), è stato alimentato con una tensione duale di $\mathrm{\pm\SI{10}{\volt}}$ e poi gli è stato fornito in ingresso un segnale a onda quadra (detto segnale di trigger) con duty cycle del 20\% e frequenza di \SI{100}{\hertz}. Abbiamo scelto il valore di duty cycle più piccolo che il generatore di forme d'onda può erogare perché nell'intervallo di tempo in cui l'onda quadra è bassa, il condensatore deve scaricarsi fino a $\mathrm{V_L^+}$ e quindi caricarsi fino a +\SI{0.7}{\volt}. Dato che tutto questo non avviene istantaneamente, ma ad una velocità legata a $\tau$, affinché il circuito funzioni correttamente è necessario che l'onda quadra stia al livello logico basso per un tempo sufficiente. La tensione picco-picco utilizzata è di \SI{10}{\volt}.\\ % controllare
\begin{figure}[h!]
	\centering
	\includegraphics[height=7cm]{immagini/circuito1}
	\caption{Fotografia del circuito monostabile con trigger di Schmitt realizzato in laboratorio.}
	\label{figura:circuito1}
\end{figure}
\\Nella figura \ref{figura:TEK00002} sono mostrati il segnale in ingresso $\mathrm{V_{IN}}$ (in giallo, CH1) e il segnale in uscita $\mathrm{V_{OUT}}$ (in azzurro, CH2). \`E stata effettuata anche la misura di $\mathrm{T_A}$ utilizzando i cursori dell'oscilloscopio, così da poterla confrontare con il risultato teorico.
\begin{figure}[h!]
	\centering
	\includegraphics[height=6.7cm]{immagini/TEK00002}
	\caption{Confronto di $\mathrm{V_{IN}}$ (CH1) con $\mathrm{V_{OUT}}$ (CH2) e misura di $\mathrm{T_A}$ con cursori.}
	\label{figura:TEK00002}
\end{figure}
\\Successivamente, abbiamo utilizzato le formule descritte nella sezione precedente per verificare la correttezza della misura della durata dell'impulso negativo in uscita ($\mathrm{T_A}$) ottenuta con l'oscilloscopio:
\\[4pt]\indent$\displaystyle{T_A=\tau\cdot\ln\biggl(1+\frac{R_2}{R_1}\biggr)=\SI{1.8}{m\second}\cdot\ln\biggl(1+\frac{\SI{12}{k\ohm}}{\SI{12}{k\ohm}}\biggr)=\SI{1.248}{m\second}}$
\\[4pt]\indent con $\displaystyle{\tau=R \cdot C=\SI{12}{k\ohm}\cdot\SI{150}{n\farad}=\SI{1.8}{m\second}}$
\\[4pt]La verifica è quindi soddisfatta perché l'oscilloscopio misura una durata dell'impulso di $\displaystyle{T_{A,meas}=2\cdot\SI{656}{\mu\second}=\SI{1.312}{m\second}}$, l'errore tra i due valori è circa del 4\%. 
\\ Abbiamo poi confrontato la durata dell'impulso negativo in uscita $\mathrm{T_A}$ con la durata dell'impulso negativo in ingresso $\mathrm{T_B}$, calcolata come:
\\[4pt]\indent$\displaystyle{T_B=(1-\delta)\cdot T=(1-20\%)\cdot\SI{10}{m\second}=\SI{8}{m\second}}$
\\[4pt]\indent con f = \SI{100}{\hertz} $\Rightarrow\;\displaystyle{T=\frac{1}{f}=\SI{10}{m\second}}$
\\[4pt]Come ci si aspetta dalla teoria, si ottiene che $\mathrm{T_B}\gg\mathrm{T_A}$, quindi il condensatore riesce a eseguire il ciclo di carica in tempo, di conseguenza il circuito funzionerà correttamente.\par
Terminata l'analisi sulla durata dell'impulso negativo, sono stati confrontati i segnali delle tensioni in ingresso e in uscita rispetto a quelle presenti sugli ingressi dell'OPAMP e a quella presente nel nodo $\mathrm{V_T}$. Riportiamo di seguito i grafici più significativi per l'analisi del circuito.\par
Per quanto riguarda la tensione sull'ingresso non invertente $\mathrm{V^+}$, essa varia tra $\mathrm{V_L^+}$ e $\mathrm{V_H^+}$ e, come si può notare dalla figura \ref{figura:TEK00003}, questi due valori non sono simmetrici. Questo succede perché:
\begin{itemize}
	\item se $\mathrm{V_{OUT}}$ è negativa (ovvero pari a $\mathrm{V_{SS}}$) si ha che il diodo $\mathrm{D_T}$ è spento e quindi:
	\\[4pt]$\displaystyle{\;\;\;V_L^+=\frac{V_{SS}}{2}=\frac{\SI{-10}{\volt}}{2}=\SI{-5}{\volt}}$;
	\item se $\mathrm{V_{OUT}}$ è positiva (ovvero pari a $\mathrm{V_{DD}}$) si ha che il diodo $\mathrm{D_T}$ è accesso e quindi:
	\\[4pt]$\displaystyle{\;\;\;V_H^+=\frac{V_{DD}}{3}=\frac{\SI{10}{\volt}}{3}=\SI{3.33}{\volt}}$.
\end{itemize}
\begin{figure}[h!]
	\centering
	\includegraphics[height=6.7cm]{immagini/TEK00003}
	\caption{Confronto di $\mathrm{V_{IN}}$ (CH1) con $\mathrm{V^+}$ (CH2).}
	\label{figura:TEK00003}
\end{figure}
La tensione sull'ingresso invertente $\mathrm{V^-}$, corrispondente alla tensione sul condensatore C ($\mathrm{V_C}$), è stata rappresentata nella figura \ref{figura:TEK00004e5} e presenta un andamento nel tempo simile ad un treno di impulsi negativi. La sua tensione non potrà mai superare \SI{0.7}{\volt} perché per tensioni superiori a questo valore, il diodo si accende e la corrente non fluirà più nel condensatore, ma andrà a massa attraverso il diodo. Dal grafico di sinistra, possiamo osservare che nell'intervallo di tempo in cui l'onda quadra è bassa la capacità risce a svolgere correttamente il ciclo di carica, mentre dal grafico di destra vediamo che quando la tensione ai capi della capacità scende da \SI{0.7}{\volt} a $\mathrm{V_L^+}$ l'uscita rimane bassa, successivamente ritorna alta fin quando non sopraggiunge un nuovo trigger.
\begin{figure}[h!]
	\centering
	\includegraphics[height=4.6cm]{immagini/TEK00004}
	\includegraphics[height=4.6cm]{immagini/TEK00005}
	\caption{Confronto di $\mathrm{V^-}$ (CH2) con $\mathrm{V_{IN}}$ (CH1, a sinistra) e con $\mathrm{V_{OUT}}$ (CH1, a destra).}
	\label{figura:TEK00004e5}
\end{figure}
%% TODO3: derivatore
\\Analizzamo ora le tensioni $\mathrm{V_T}$ e $\mathrm{V_{IN}}$. Il sottocircuito a sinistra del diodo $\mathrm{D_T}$ si comporta come un derivatore della tensione in ingresso, ma solo se lavoriamo ad una frequenza molto minore della frequenza di taglio. La frequenza di taglio del filtro è:
\\[4pt]\indent$\displaystyle{f_T = \frac{1}{2\pi\cdot R_T\cdot C_T} = \frac{1}{2\pi\cdot \SI{12}{k\ohm}\cdot\SI{1}{n\farad}} =\SI{13.263}{k\hertz}}$
\\[8pt]Dato che il nostro circuito lavora ad una frequenza di \SI{100}{\hertz}, la condizione $f_T\gg f$ è rispettata, pertanto al nodo $\mathrm{V_T}$ avremo la derivata della tensione del nodo $\mathrm{V_{IN}}$. Quando l'onda quadra effettua una transizione verso il basso, avremo un impulso negativo, mentre se effettua una transizione verso l'alto, avremo un impulso positivo la cui tensione massima è però troncata dall'accensione del diodo. Se l'onda quadra è costante, la sua derivata è nulla, perciò la tensione del nodo $\mathrm{V_T}$ è anch'essa nulla. In figura \ref{figura:TEK00013} è mostrato l'andamento nel tempo delle due tensioni considerate. 
\begin{figure}[h!]
	\centering
	\includegraphics[height=6.7cm]{immagini/TEK00013}
	\caption{Confronto di $\mathrm{V_{IN}}$ (CH1) con $\mathrm{V_T}$ (CH2).}
	\label{figura:TEK00013}
\end{figure}
%Se confrontiamo le tensioni all'ingresso non invertente ed in uscita, vediamo che entrambe hanno lo stesso andamento, ma la tensione $\mathrm{V_{OUT}}$ è limitata dalle alimentazioni mentre la tensione $\mathrm{V^+}$ è limitata dalle due soglie del trigger di Schmitt, $\mathrm{V_L^+}$ e $\mathrm{V_H^+}$.
%%%
\\Nella figura successiva, sono mostrate tutte le tensioni che abbiamo misurato in precedenza, ottenute simulando il circuito su un periodo di tempo dell'onda quadra. Anche su questo grafico possiamo ritrovare tutte le considerazioni fatte per i grafici precedenti, ad eccezione dell'istante iniziale, perché il simulatore inizializza i nodi $\mathrm{V^-}$ e $\mathrm{V_{OUT}}$ a \SI{0}{\volt}. 
\begin{figure}[h!]
	\centering
	\includegraphics[height=6.5cm]{immagini/plot_sim_2} 
	\caption{Confronto di $\mathrm{V_{IN}}$, $\mathrm{V_{OUT}}$, $\mathrm{V^+}$ e $\mathrm{V^-}$.}
	\label{figura:simulazione}
\end{figure}
\\Infine, è stato considerato lo stato stabile di questo circuito, che consiste nell'aggiungere al circuito del precedente laboratorio soltanto il diodo D ottenendo il circuito in figura \ref{figura:schema1stabile}.  Questo circuito è in pratica il monostabile appena analizzato, a cui togliamo però la rete che realizza i trigger. Per quest'analisi sono stati mantenuti invariati i valori dei componenti utilizzati. Inoltre, al circuito risultante non si applica nessun segnale in ingresso.
\begin{figure}[h!]
	\centering
	\includegraphics[height=7cm]{immagini/schema1stabile}
	\caption{Schema del circuito nello stato stabile.}
	\label{figura:schema1stabile}
\end{figure}
\\Analizzando il segnale sull'ingresso invertente dell'OPAMP, $\mathrm{V^-}$, si può notare che questa tensione presenta una fase iniziale di carica e poi, una volta raggiunti gli \SI{0.7}{\volt}, viene bloccata poichè è stato raggiunto il valore della tensione ai capi del diodo e quindi esso si accende, impedendo alla capacità di continuare a caricarsi (la corrente che prima fluiva nel condensatore e lo caricava, ora andrà a massa attraverso il diodo acceso). Invece il segnale in uscita, $\mathrm{V_{OUT}}$, rimane al valore dell'alimentazione positiva $\mathrm{V_{DD}}$ e quindi non riesce a commutare, siccome $\mathrm{V^-}$ non riesce mai a superare la soglia $\mathrm{V_H^+}$. Quindi, questo circuito rimane nello stato stabile e  non oscilla. Il grafico della simulazione è riportato in figura \ref{figura:simulazione1}. 
\begin{figure}[h!]
	\centering
	\includegraphics[height=4.6cm]{immagini/plot_sim_stabile}
	\caption{Confronto di $\mathrm{V^-}$ (CH1) con $\mathrm{V_{OUT}}$ (CH2) nello stato stabile.}
	\label{figura:simulazione1}
\end{figure}
\\Per poter rendere questo circuito monostabile, è necessario che la soglia $\mathrm{V_H^+}$ risulti inferiore rispetto a $\mathrm{V^-}$, in modo da avere una transizione su $\mathrm{V_{OUT}}$. Per avere questo effetto è necessario collegare sull'ingresso non invertente dell'OPAMP la rete che funziona come derivatore.
\newpage
\section{Circuito 2: circuito monostabile con NE555}
\subsection{Schema del circuito e Funzione di Trasferimento}
Un circuito monostabile può anche essere realizzato utilizzando un circuito integrato, il timer 555. Nella figura seguente (figura \ref{figura:schema2}) è mostrato il package del NE555 con i nomi e la funzione di ogni terminale, oltre che alle connessioni fra i vari componenti.
\begin{figure}[h!]
	\centering
	\includegraphics[height=7.5cm]{immagini/schema2}
	\caption{Schema del circuito monostabile con NE555.}
	\label{figura:schema2}
\end{figure}
\\In questa modalità, quando al pin 2 viene applicato un trigger negativo (che fa passare la tensione di questo pin dal livello logico alto a quello basso), l'uscita da bassa rimane alta per un periodo di tempo che dipende dalla rete collegata ai pin 6 e 7, dopodichè ritorna bassa e rimane a questo valore fin quando non arriva un nuovo trigger.\par
L'intervallo di tempo in cui l'uscita rimane alta è dettato dal processo di carica del condensatore $\mathrm{C_1}$: quando arriva il trigger negativo sul pin 2, l'uscita del NE555, corrispondente al pin 3, diventa alta e il condensatore $\mathrm{C_1}$ si carica attraverso la resistenza $\mathrm{R}$. Il processo di carica continua fin quando la tensione ai capi della capacità raggiunge la tensione di $\displaystyle{2/3\;V_{DD}}$, questo valore è determinato dallo schema interno del NE555. Raggiunta questa tensione, l'uscita del NE555 ritorna al livello logico basso e la capacità si scarica istantaneamente a massa, perciò il circuito ritorna autonomamente al suo stato stabile.
\\La durata dell'impulso in uscita al NE555 si ricava dalla formula di carica del condensatore: 
\\[4pt]\indent$\displaystyle{v_C=V_{CC}\cdot\bigl(1-e^{-\frac{t}{R\cdot C_1}}\bigr)\indent\rightarrow\indent\frac{2}{3}\cdot V_{CC}=V_{CC}\cdot\bigl(1-e^{-\frac{t}{R\cdot C_1}}\bigr)}$
\\[4pt]\indent$\displaystyle{\Rightarrow\indent T=-R\cdot C_1\cdot\ln\frac{1}{3}=1.1\cdot R\cdot C_1}$
\\[4pt]La capacità $\mathrm{C_2}$ ha il solo scopo di ripulire il segnale di massa da eventuali disturbi ad alta frequenza.
\subsection{Analisi e dati sperimentali}
%R ... k\ohm --> 12k\ohm
%R 4.698 k\ohm --> 4.7 k\ohm
Per simulare il trigger sul pin 2, utilizziamo un'onda quadra di ampiezza picco-picco di \SI{10}{\volt} e offset DC di \SI{5}{\volt}. Il duty cycle utilizzato è di 80\%, ovvero il valore più alto fornito dallo strumento, perché è necessario che i trigger negativi siano distanziati di un tempo maggiore rispetto alla durata dell'impulso alto in uscita. Anche la frequenza sarà fatta variare in modo tale da rispettare sempre questa condizione, altrimenti il circuito non funziona correttamente. In figura \ref{figura:circuito2} vediamo la fotografia del circuito con la resistenza R da \SI{12}{k\ohm}, $\mathrm{C_1}$ da  \SI{150}{n\farad} e $\mathrm{C_2}$ da  \SI{1}{n\farad}. Il modello di timer 555 utilizzato è il LM555, alimentiamo la breadboard fra massa e +\SI{10}{\volt} perché questo componente funziona ad alimentazione singola.
\begin{figure}[h!]
	\centering
	\includegraphics[height=7cm]{immagini/circuito2}
	\caption{Fotografia del circuito monostabile con LM555 realizzato in laboratorio.}
	\label{figura:circuito2}
\end{figure}
%%%
\\Per verificare la relazione esistente fra la durata dell'impulso T ed i valori della resistenza e della capacità, andiamo a misurare la durata dell'impulso al variare di R, con $\mathrm{C_1}$ fissata a \SI{150}{n\farad}, successivamente fissiamo la capacità ad un nuovo valore, \SI{10}{n\farad}, e facciamo variare nuovamente la resistenza. Le misure delle resistenze scelte sono riportate in tabella \ref{table:mis_res2}, mentre quelle dell'impulso T al variare della resistenza per i due valori di capacità sono riportate nelle tabelle \ref{table:impulsoRvar} e \ref{table:impulsoCvar}.
\begin{table}[h!]
	\centering
	\begin{tabular}{|c|c|}
	\hline
		\textbf{Valore nominale} & \textbf{Valore misurato}\\ 
		\hline
		\SI{4.6}{k\ohm} & \SI{4.625}{k\ohm} \\ 
		\hline
		 \SI{12}{k\ohm} & \SI{11.898}{k\ohm} \\ 
		\hline
		\SI{18}{k\ohm} & \SI{17.998}{k\ohm} \\ 
		\hline
		 \SI{22}{k\ohm} & \SI{23.947}{k\ohm} \\ 
		\hline
	\end{tabular}
	\caption{Misure delle resistenze utilizzate per il circuito monostabile con NE555.}
	\label{table:mis_res2}
\end{table}
\begin{table}[h!]
	\centering
	\begin{tabular}{|c|c|c|c|}
		\cline{2-4} 
		\multicolumn{1}{c|}{} & \textbf{Valore calcolato} & \textbf{Valore misurato}& \textbf{Errore percentuale}\\ 
		\hline
		$\mathbf{{R = 4.6k}\Omega}$ & \SI{763}{\mu\second} & \SI{772}{\mu\second} & 1.16\% \\ 
		\hline
		$\mathbf{{R = 12k}\Omega}$ & \SI{1.96}{m\second} & \SI{1.99}{m\second} & 1.37\% \\ 
		\hline
		$\mathbf{{R = 18k}\Omega}$ & \SI{2.97}{m\second} & \SI{3.00}{m\second} & 1.02\% \\ 
		\hline
		$\mathbf{{R = 22k}\Omega}$ & \SI{3.95}{m\second} & \SI{4.00}{m\second} & 1.23\% \\ 
		\hline
	\end{tabular}
	\caption{Misura e valore teorico della durata dell'impulso al variare della resistenza, con $\mathrm{C_1}$=\SI{150}{n\farad}.}
	\label{table:impulsoRvar}
\end{table}
\begin{table}[h!]
	\centering
	\begin{tabular}{|c|c|c|c|}
		\cline{2-4} 
		\multicolumn{1}{c|}{} & \textbf{Valore calcolato} & \textbf{Valore misurato}& \textbf{Errore percentuale}\\ 
		\hline
		$\mathbf{{R = 4.6k}\Omega}$ & \SI{50.9}{\mu\second} & \SI{53.2}{\mu\second} & 4.57\% \\ 
		\hline
		$\mathbf{{R = 12k}\Omega}$ & \SI{131}{\mu\second} & \SI{135}{\mu\second} & 3.15\% \\ 
		\hline
		$\mathbf{{R = 18k}\Omega}$ & \SI{198}{\mu\second} & \SI{207}{\mu\second} & 4.56\% \\ 
		\hline
		$\mathbf{{R = 22k}\Omega}$ & \SI{263}{\mu\second} & \SI{270}{\mu\second} & 2.50\% \\ 
		\hline
	\end{tabular}
	\caption{Misura e valore teorico della durata dell'impulso al variare della resistenza, con $\mathrm{C_1}$=\SI{10}{n\farad}.}
	\label{table:impulsoCvar}
\end{table}
\newpage
\noindent Le misure di T ottenute con le due diverse capacità sono confrontabili con i valori teorici, nel secondo caso le misure sono affette da un errore maggiore perché la capacità utilizzata è più piccola, quindi l'onda quadra resta alta per un periodo di tempo minore, di  conseguenza gli errori di misura impattano maggiormente sui risultati ottenuti. \par
Nella figura seguente (figura \ref{figura:TEK00014}) è mostrato il grafico in uscita all'oscilloscopio per $R = \SI{12}{k\ohm}$ e $C_1 = \SI{150}{n\farad}$, in cui possiamo vedere la misura di T effettuata dall'oscilloscopio. Il grafico è simile per le altre combinazioni di $R$ e $C_1$, cambia solo la durata dell'intervallo di tempo T. Dal grafico si nota anche che quando la tensione del pin 2 ha una transizione negativa, l'uscita si porta alta per un intervallo di tempo di durata T, successivamente rimane al livello logico basso fin quando non si ha una nuova transizione negativa del segnale in ingresso.
\begin{figure}[h!]
	\centering
	\includegraphics[height=6.5cm]{immagini/TEK00014}
	\caption{Confronto di fra la tensione al pin 2 (CH1) e la tensione al pin 3 (CH2).}
	\label{figura:TEK00014}
\end{figure}
\\ Per visualizzare meglio i dati delle tabelle, confrontiamo il grafico della durata dell'impulso misurata con quella teorica al variare della resistenza, per entrambe le capacità. Il risultato di quest'elaborazione è mostrato in figura \ref{figura:graf1e2}.\begin{figure}[h!]
	\centering
	\includegraphics[height=5.8cm]{immagini/graficomis1}
	\includegraphics[height=5.8cm]{immagini/graficomis2}
	\caption{Grafico di T in funzione di R per $\mathrm{C_1}$ = \SI{150}{n\farad} (a sinistra) e $\mathrm{C_1}$ = \SI{10}{n\farad} (a destra).}
	\label{figura:graf1e2}
\end{figure}
\\ In entrambi i casi, le misure sono sovrapponibili ai risultati teorici. Come ci aspettiamo dalla formula, l'andamento di T in funzione di R è di tipo lineare, infatti a resistenze di valore maggiore corrispondono durate dell'impulso T maggiori. \par
Per analizzare l'influenza della capacità su T, possiamo andare a confrontare fra di loro i grafici delle misure per i due valori di capacità. Questo grafico è mostrato in figura \ref{figura:confrontoplot}.
\begin{figure}[h!]
	\centering
	\includegraphics[height=6.5cm]{immagini/graficoconfronto1}
	\caption{Confronto fra i grafici delle diverse misure con i due valori di capacità.}
	\label{figura:confrontoplot}
\end{figure}
\\Anche in questo caso possiamo dire che la capacità e l'intervallo T sono proporzionali, perché a valori di capacità minori corrispondono durate dell'impulso minori, viceversa per capacità maggiori la durata di T è maggiore. % pendenza retta ?

%----------------------------------------------------------------------------------------

\end{document}
